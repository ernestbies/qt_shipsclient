\subsection*{}

Autorzy\+: Ernest Bieś, Konrad Czechowski, Dawid Kwaśny





\section*{Podstawowe informacje }

\char`\"{}\+Statki kosmiczne\char`\"{} to gra logiczna będąca połączeniem dwóch popularnych gier \char`\"{}\+Statki\char`\"{} oraz \char`\"{}\+Saper\char`\"{}.

\section*{Zasady gry }

Po rozpoczęciu gry na planszy o wymiarach 9x9 utworzonej z kwadratów są losowo rozmieszczone statki kosmiczne o różnych wielkościach\+: Dwa {\itshape Transportowce} o wymiarze 4 Trzy {\itshape Samoloty kosmiczne} o wymiarze 3 Trzy {\itshape Wahadłowce} o wymiarze 2 Dwa {\itshape Szturmowce} o wymiarze 1

Statki nie są widoczne. Zadaniem gracza jest odnalezienie wszystkich statków w jak najmniejszej liczbie kroków. W przypadku odkrycia pola, które nie zawiera żadnego statku widoczna jest liczba, która wskazuje na ilu polach sąsiadujących z odkrytym polem znajdują się statki. Wobec powyższego gracz musi podejmować swoją decyzję dotyczącą pola, które w następnej kolejności odkryje na podstawie logicznego myślenia, gdyż tym samym zwiększa swoje szanse na odnalezienie statków w pobliżu.

\section*{Opis klienta }

\subsection*{Interfejs graficzny klienta }



W głównym oknie programu po lewej stronie widoczna jest plansza o rozmiarze 9x9 kwadratów na której wyświetlane są ikony zestrzelonych statków oraz pola, które zostały odkryte przez gracza, a nie zawierają żadnych statków. Po uruchomieniu gry po lewej stronie wyświetlone zostają podstawowe informacje dotyczące zasad gry oraz statków jakie musimy zestrzelić. W górnym lewym rogu znajduje się przycisk za pomocą którego możemy ukryć zasady gry. W lewym dolnym rogu okna znajdują się pola służące do wprowadzenia nazwy użytkownika i hasła oraz przyciski za pomocą których możemy utworzyć nową grę na serwerze lub wczytać już istniejącą.\+Na dole w srodku okna znajduje się konsola na której wyświetlane są informacje dotyczące akcji podejmowanych przez gracza. Z prawej strony w dolnej części okna znajduje się licznik z aktualnie wykonanymi krokami przez gracza.



W trakcie gry użytkownik po wskazaniu myszką wybranego pola i naciśnięciu lewego przycisku myszki odsłania poszczególne pola na planszy. W przypadku trafienia statku jego ikona zostaje wyświetlona w odpowiednim miejscu na planszy, natomiast w przypadku \char`\"{}pudła\char`\"{} we wskazanym polu pojawia się znak X oraz liczba wskazująca na ilu polach sąsiadujących ze wskazanym znajdują się statki. Równocześnie po wskazaniu pola odtwarzany jest odpowiedni dźwięk.

\subsection*{Instrukcja obsługi }

Po uruchomieniu klienta użytkownik wprowadza nazwę użytkownika oraz hasło, a następnie naciska przycisk \char`\"{}\+Nowa gra\char`\"{}. Zostaje utworzona nowa gra dla użytkownika, następnie korzystając z myszki naciska lewy przycisk na wybranym polu na planszy. Jeżeli pod odkrytym polem znajduje się statek, jego ikona zostanie wyświetlona w odpowiednim kwadracie oraz program zasygnalizuje trafienie odpowiednim dźwiękiem. Jeżeli natomiast wskazane pole nie zawiera żadnego statku to zostanie na nim wyświetlony znak X oraz liczba wskazująca na ilu polach sąsiadujących znajdują się statki. Jednocześnie za każdym wskazaniem pola zostaje zwiększony licznik znajdujący się w prawym, dolnym roku ekranu. Jednocześnie na konsoli będą pojawiały się komunikaty dotyczące trafienia danego statku, jego nazwy lub też, że oddany strzał to niestety \char`\"{}pudło\char`\"{}. Po zestrzeleniu wszystkich statków kosmicznych wyświetlany jest komunikat z zapytaniem czy użytkownik chce zakończyć działanie programu. W przypadku udzielenia negatywnej odpowiedzi gra jest zapisywana na serwerze, natomiast klient zeruje wszelkie dane i ustawienia celem umożliwienia użytkownikowi rozpoczęcia nowej gry.

 